\documentclass[11pt]{article}
\usepackage{fancyvrb}
\usepackage{comment}
\usepackage[paper=letterpaper,margin=1.0in,includehead=false,includefoot=false]{geometry}

\newcommand{\cursor}{%
\begin{center}
\marginpar{\vskip 4pt\sc Cursor}
\begin{tabular*}{\linewidth}{c}
\hline
\end{tabular*}
\end{center}
}

\DefineVerbatimEnvironment{Code}{Verbatim}{fontsize=\footnotesize,xleftmargin=5pt}
\DefineVerbatimEnvironment{SemiCode}{Verbatim}{fontsize=\small,commandchars=\+\{\}}

\begin{document}

\title{Domain Specific Languages}
\author{Andy Gill, University of Kansas}
\maketitle

\section{A Domain Specific Language}

There are many ways to give a computer instructions.
%
%% "might" might sound better than "may"
An electrical engineer might write a MATLAB program,
a database administrator may write an SQL script,
a hardware engineer may write in Verilog,
and an accountant may write a spreadsheet
with embedded formulas.
%
%% This sentence isn't clear, nor does it lead into the next sentence very well.
%% I'd suggest expanding on this point to make it clearer.
Aside from the difference in language, there is an
important difference in {\em form\/} and {\em idiom\/}.
%% Maybe you should emphasise "idiom" as well?
%
All of these examples use languages
customized to the job at hand, and build computational
requests in a form both familiar and productive
for the programmer (though an accountant may
not think of herself as a programmer.)
All these examples are use of domain specific languages.


A Domain Specific Language is a special purpose language,
designed to encapsulate possible computations in a specific
domain. Following our earlier examples of MATLAB, SQL,
%% earlier examples of?
Verilog, and spreadsheets, the domains would be scientific modeling,
database queries and updates, hardware circuits, and financial computations, respectively.
Considering SQL specifically, there is nothing SQL does that could not
be done in Java or C, or any other general purpose programming
language. SQL simply bundles the actions needed to
interact with a database into a useable language,
%% is useable the correct adjective here?  Java and C are useable languages too.
and the language becomes the interface to communicate requests
to the database engine.
To pick a concrete example,
consider the act of trying to list a dynamically updated leader-board
for an online programming contest~\cite{Gill:11:CatchingSat}:
\begin{Code}
SELECT ROUND(SUM(s.Score)) as ss, t.TeamName FROM Solution s -- an aggregate score and team name
   LEFT JOIN Team t ON SolutionTeam = TeamId                 -- where the team has the correct id
   GROUP BY s.SolutionTeam                                   -- grouped by team name
   ORDER BY ss DESC                                          -- ordered by score
   LIMIT 20                                                  -- returning the first 20
\end{Code}
In a handful on lines, this query performs a complex
database search, and some computation, finding
the names and aggregate scores of the top 20 teams.

There are two classes of DSL. (MORE)

((Figure 1.))

In this paper, we will examine Embedded DSL, focusing on 

\section{Embeded DSLs}

((EXAMPLE of DSLs))

Just a way of thinking about a library.


What is an embedded DSL? We introduce it by looking
at the history of $\mu$FP. Around 10 years after
$\mu$FP, a language called Ruby emerged~\cite{Hutton:92:PhD,Hutton:93:RubyInterp};
a $\mu$FP based on relations, with an even richer algebra,
and at least one of the Ruby variants had a
concrete DSL syntax, and could generate small circuits.
Unfortunately, the rich algebras of relations
were technically challenging to work with.
Two independent Ruby implementations
required a considerable maintenance of the DSL compiler,
and both systems were later abandoned.

The cost of maintaining a DSL outweighed the research
discoveries being made. However, at this time,
the idea of an EDSL~\cite{Hudak:96:BuildingEDSLs,Elliott:03:CompileDSEL-JFP} emerged.
%
An EDSL is a library in a host language
that has the look, feel and semantics of its own language,
customized to a specific problem domain.
Haskell~\cite{Haskell98Book}, our functional language of choice, is a great host for EDSLs.
%% Maybe these two sentences should be merged with a ``because of'' in the middle.
Flexible overloading, a powerful type system, and lazy semantics facilitate this.

The functional programming community took the ideas
of EDSLs, which considerably lowered the costs of developing
and maintaining a DSL, and revisited the $\mu$FP ideas,
but in the context this time of functions.
%% Should that be suffixed?
This line of work, often suffixed with the word Lava,
provided a number of implementations~\cite{Claessen:01:PhD,Singh:04:ReconfigLava,Sander:03:ForSyDe,ODonnell:02:Hydra}.
Being based on functions rather than relations,
they had a more straightforward algebras
of rules. EDSLs were less expensive to maintain, and the research blossomed.
But no system supported rewrites based on the published properties.

Another observation to make regarding EDSLs is that
there are two flavors:
%% I don't like the use of ``First'' and ``Second'' in these bullet points.
\begin{itemize}
\item First, {\em DSLs that use a shallow embedding\/}, where values are computed with directly.
The result of a computation in a shallow DSL is a value.
\item Second, {\em DSLs that use a deep embedding\/} build an abstract syntax tree~\cite{Elliott:03:CompileDSEL-JFP}.
The result of a computation inside a deep DSL
is a structure, not a value, and this structure can be used to compute a value,
or be cross-compiled before being evaluated.
\end{itemize}
This is an important distinction. Historically, EDSLs have been shallow,
and as such are simply a ways of structuring an API to a library. Deep EDSLs
have the ability to {\em stage\/} code, that is executing a program
can generate another program, much like the well-known YACC DSL,
but at the cost of significantly restricting what forms of the DSL can
generate valid output from.
%% That sentence is a bit garbled.

To allow us to be more specific, we now
look at two case-studies:
\begin{itemize}
\item {\bf Kansas Lava} -- recent work by the PI on a hardware description DSL;
\item and {\bf Sunroof} -- a DSL for calling and creating JavaScript, also by the PI.
%% sunroof is your work too, right?
\end{itemize}
The unifying theme is that embedded DSLs can
be pragmatic, productive and useful. We present
these to allow us to make a technical argument that adding rewrite systems to
DSLs will improve their usefulness by removing
some of the inherent and outstanding shortcomings.

\section{Deep Embeddings of DSLs}

\section{Reification Tricks}

The process of extracting a structure from a deeply embedded ESDL
is called reification. 


\cursor{}

 * running the description to get the AST 

\section{Use of Haskell’s type system for various things }

 * expression data type 
 * the phantom types trick 
 * Use of type classes (e.g. distinction of Seq and Comb in Signal) 
 * usefulness of laziness 
 * smart constructors ? 
 * sharing ?

%--------------------------------
% CAREER12


\section{Domain Specific Languages}








\end{document}
